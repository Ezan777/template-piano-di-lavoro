%----------------------------------------------------------------------------------------
%   USEFUL COMMANDS
%----------------------------------------------------------------------------------------

\newcommand{\dipartimento}{Dipartimento di Matematica ``Tullio Levi-Civita''}

%----------------------------------------------------------------------------------------
% 	USER DATA
%----------------------------------------------------------------------------------------

% Data di approvazione del piano da parte del tutor interno; nel formato GG Mese AAAA
% compilare inserendo al posto di GG 2 cifre per il giorno, e al posto di 
% AAAA 4 cifre per l'anno
\newcommand{\dataApprovazione}{Data}

% Dati dello Studente
\newcommand{\nomeStudente}{Enrico}
\newcommand{\cognomeStudente}{Zangrando}
\newcommand{\matricolaStudente}{2000547}
\newcommand{\emailStudente}{enrico.zangrando@studenti.unipd.it}
\newcommand{\telStudente}{+ 39 348 68 51 731}

% Dati del Tutor Aziendale
\newcommand{\nomeTutorAziendale}{Jonathan}
\newcommand{\cognomeTutorAziendale}{Volpato}
\newcommand{\emailTutorAziendale}{jonathan.volpato@vimar.com}
\newcommand{\telTutorAziendale}{+ 39 042 44 88 470}
\newcommand{\ruoloTutorAziendale}{}

% Dati dell'Azienda
\newcommand{\ragioneSocAzienda}{Vimar S.p.A}
\newcommand{\indirizzoAzienda}{Viale Vicenza 14, Marostica (VI)}
\newcommand{\sitoAzienda}{http://vimar.com}
\newcommand{\emailAzienda}{}
\newcommand{\partitaIVAAzienda}{P.IVA 02161730243}

% Dati del Tutor Interno (Docente)
\newcommand{\titoloTutorInterno}{Prof.}
\newcommand{\nomeTutorInterno}{Claudio Enrico}
\newcommand{\cognomeTutorInterno}{Palazzi}

\newcommand{\prospettoSettimanale}{
     % Personalizzare indicando in lista, i vari task settimana per settimana
     % sostituire a XX il totale ore della settimana
    \begin{itemize}
        \item \textbf{Prima Settimana (40 ore)}
        \begin{itemize}
            \item Incontro con persone coinvolte nel progetto per discutere i requisiti e le richieste
            relativamente al sistema da sviluppare;
            \item Verifica credenziali e strumenti di lavoro assegnati;
            \item Presa visione dell’infrastruttura esistente;
            \item Formazione sulle tecnologie adottate;
        \end{itemize}
        \item \textbf{Seconda Settimana - Sottotitolo (40 ore)} 
        \begin{itemize}
            \item Ricerca e formazione sul protocollo utilizzato per la comunicazione mobile-smartwatch;
        \end{itemize}
        \item \textbf{Terza Settimana - Sottotitolo (40 ore)} 
        \begin{itemize}
            \item Progettazione dell'app per smartwatch in modo che possa interfacciarsi con l'app esistente;
        \end{itemize}
        \item \textbf{Quarta Settimana - Sottotitolo (40 ore)} 
        \begin{itemize}
            \item Coding UX/UI;
            \item Unit test e conseguente bug fixing;
        \end{itemize}
        \item \textbf{Quinta Settimana - Sottotitolo (40 ore)} 
        \begin{itemize}
            \item Coding UX/UI;
            \item Unit test e conseguente bug fixing;
        \end{itemize}
        \item \textbf{Sesta Settimana - Sottotitolo (40 ore)} 
        \begin{itemize}
            \item Coding UX/UI;
            \item Unit test e conseguente bug fixing;
        \end{itemize}
        \item \textbf{Settima Settimana - Sottotitolo (40 ore)} 
        \begin{itemize}
            \item Coding UX/UI;
            \item Unit test e conseguente bug fixing;
        \end{itemize}
        \item \textbf{Ottava Settimana - Conclusione (40 ore)} 
        \begin{itemize}
            \item Coding UX/UI;
            \item Documentazione;
            \item Presentazione/dimostrazione del progetto svolto;
            \item Possibile integrazione di Google Assistant;
        \end{itemize}
    \end{itemize}
}

% Indicare il totale complessivo (deve essere compreso tra le 300 e le 320 ore)
\newcommand{\totaleOre}{320}

\newcommand{\obiettiviObbligatori}{
	 \item \underline{\textit{O01}}: comprensione delle tecnologie utilizzate in Vimar;
	 \item \underline{\textit{O02}}: creazione di un app per smartwatch che possa comunicare con un'app companion mobile;
	 \item \underline{\textit{O03}}: integrazione della precedente app con un sistema IoT vimar;
	 
}

\newcommand{\obiettiviDesiderabili}{
	 \item \underline{\textit{D01}}: comprensione sul funzionamento di Google Assistant e come integrarlo in app;
}

\newcommand{\obiettiviFacoltativi}{
	 \item \underline{\textit{F01}}: integrazione di Google Assistant;
}